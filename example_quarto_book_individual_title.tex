% Options for packages loaded elsewhere
\PassOptionsToPackage{unicode}{hyperref}
\PassOptionsToPackage{hyphens}{url}
%
\documentclass[
  11pt,
]{scrbook}

\usepackage{amsmath,amssymb}
\usepackage{setspace}
\usepackage{iftex}
\ifPDFTeX
  \usepackage[T1]{fontenc}
  \usepackage[utf8]{inputenc}
  \usepackage{textcomp} % provide euro and other symbols
\else % if luatex or xetex
  \usepackage{unicode-math}
  \defaultfontfeatures{Scale=MatchLowercase}
  \defaultfontfeatures[\rmfamily]{Ligatures=TeX,Scale=1}
\fi
\usepackage{lmodern}
\ifPDFTeX\else  
    % xetex/luatex font selection
\fi
% Use upquote if available, for straight quotes in verbatim environments
\IfFileExists{upquote.sty}{\usepackage{upquote}}{}
\IfFileExists{microtype.sty}{% use microtype if available
  \usepackage[]{microtype}
  \UseMicrotypeSet[protrusion]{basicmath} % disable protrusion for tt fonts
}{}
\makeatletter
\@ifundefined{KOMAClassName}{% if non-KOMA class
  \IfFileExists{parskip.sty}{%
    \usepackage{parskip}
  }{% else
    \setlength{\parindent}{0pt}
    \setlength{\parskip}{6pt plus 2pt minus 1pt}}
}{% if KOMA class
  \KOMAoptions{parskip=half}}
\makeatother
\usepackage{xcolor}
\usepackage[left = 25mm, right = 25mm, top = 30mm, bottom =
30mm]{geometry}
\setlength{\emergencystretch}{3em} % prevent overfull lines
\setcounter{secnumdepth}{5}
% Make \paragraph and \subparagraph free-standing
\ifx\paragraph\undefined\else
  \let\oldparagraph\paragraph
  \renewcommand{\paragraph}[1]{\oldparagraph{#1}\mbox{}}
\fi
\ifx\subparagraph\undefined\else
  \let\oldsubparagraph\subparagraph
  \renewcommand{\subparagraph}[1]{\oldsubparagraph{#1}\mbox{}}
\fi


\providecommand{\tightlist}{%
  \setlength{\itemsep}{0pt}\setlength{\parskip}{0pt}}\usepackage{longtable,booktabs,array}
\usepackage{calc} % for calculating minipage widths
% Correct order of tables after \paragraph or \subparagraph
\usepackage{etoolbox}
\makeatletter
\patchcmd\longtable{\par}{\if@noskipsec\mbox{}\fi\par}{}{}
\makeatother
% Allow footnotes in longtable head/foot
\IfFileExists{footnotehyper.sty}{\usepackage{footnotehyper}}{\usepackage{footnote}}
\makesavenoteenv{longtable}
\usepackage{graphicx}
\makeatletter
\def\maxwidth{\ifdim\Gin@nat@width>\linewidth\linewidth\else\Gin@nat@width\fi}
\def\maxheight{\ifdim\Gin@nat@height>\textheight\textheight\else\Gin@nat@height\fi}
\makeatother
% Scale images if necessary, so that they will not overflow the page
% margins by default, and it is still possible to overwrite the defaults
% using explicit options in \includegraphics[width, height, ...]{}
\setkeys{Gin}{width=\maxwidth,height=\maxheight,keepaspectratio}
% Set default figure placement to htbp
\makeatletter
\def\fps@figure{htbp}
\makeatother
\newlength{\cslhangindent}
\setlength{\cslhangindent}{1.5em}
\newlength{\csllabelwidth}
\setlength{\csllabelwidth}{3em}
\newlength{\cslentryspacingunit} % times entry-spacing
\setlength{\cslentryspacingunit}{\parskip}
\newenvironment{CSLReferences}[2] % #1 hanging-ident, #2 entry spacing
 {% don't indent paragraphs
  \setlength{\parindent}{0pt}
  % turn on hanging indent if param 1 is 1
  \ifodd #1
  \let\oldpar\par
  \def\par{\hangindent=\cslhangindent\oldpar}
  \fi
  % set entry spacing
  \setlength{\parskip}{#2\cslentryspacingunit}
 }%
 {}
\usepackage{calc}
\newcommand{\CSLBlock}[1]{#1\hfill\break}
\newcommand{\CSLLeftMargin}[1]{\parbox[t]{\csllabelwidth}{#1}}
\newcommand{\CSLRightInline}[1]{\parbox[t]{\linewidth - \csllabelwidth}{#1}\break}
\newcommand{\CSLIndent}[1]{\hspace{\cslhangindent}#1}

\usepackage{scrlayer-scrpage}
\pagestyle{scrheadings}

\ohead{\headmark} % Überschrift 1 im äußeren Kopf
\automark[chapter]{chapter} % Kapitel auf beiden Seiten im header
\setheadsepline{0.4pt} % Linie unter der Kopfzeile
\setfootsepline{0.4pt} % Linie über der Fußzeile
\makeatletter
\makeatother
\makeatletter
\makeatother
\makeatletter
\@ifpackageloaded{caption}{}{\usepackage{caption}}
\AtBeginDocument{%
\ifdefined\contentsname
  \renewcommand*\contentsname{Inhaltsverzeichnis}
\else
  \newcommand\contentsname{Inhaltsverzeichnis}
\fi
\ifdefined\listfigurename
  \renewcommand*\listfigurename{Abbildungsverzeichnis}
\else
  \newcommand\listfigurename{Abbildungsverzeichnis}
\fi
\ifdefined\listtablename
  \renewcommand*\listtablename{Tabellenverzeichnis}
\else
  \newcommand\listtablename{Tabellenverzeichnis}
\fi
\ifdefined\figurename
  \renewcommand*\figurename{Abbildung}
\else
  \newcommand\figurename{Abbildung}
\fi
\ifdefined\tablename
  \renewcommand*\tablename{Tabelle}
\else
  \newcommand\tablename{Tabelle}
\fi
}
\@ifpackageloaded{float}{}{\usepackage{float}}
\floatstyle{ruled}
\@ifundefined{c@chapter}{\newfloat{codelisting}{h}{lop}}{\newfloat{codelisting}{h}{lop}[chapter]}
\floatname{codelisting}{Listing}
\newcommand*\listoflistings{\listof{codelisting}{Listingverzeichnis}}
\makeatother
\makeatletter
\@ifpackageloaded{caption}{}{\usepackage{caption}}
\@ifpackageloaded{subcaption}{}{\usepackage{subcaption}}
\makeatother
\makeatletter
\@ifpackageloaded{tcolorbox}{}{\usepackage[skins,breakable]{tcolorbox}}
\makeatother
\makeatletter
\@ifundefined{shadecolor}{\definecolor{shadecolor}{rgb}{.97, .97, .97}}
\makeatother
\makeatletter
\makeatother
\makeatletter
\makeatother
\ifLuaTeX
\usepackage[bidi=basic]{babel}
\else
\usepackage[bidi=default]{babel}
\fi
\babelprovide[main,import]{ngerman}
% get rid of language-specific shorthands (see #6817):
\let\LanguageShortHands\languageshorthands
\def\languageshorthands#1{}
\ifLuaTeX
  \usepackage{selnolig}  % disable illegal ligatures
\fi
\IfFileExists{bookmark.sty}{\usepackage{bookmark}}{\usepackage{hyperref}}
\IfFileExists{xurl.sty}{\usepackage{xurl}}{} % add URL line breaks if available
\urlstyle{same} % disable monospaced font for URLs
\hypersetup{
  pdftitle={Titel},
  pdfauthor={Max Mustermann; Maria Musterfrau},
  pdflang={de},
  hidelinks,
  pdfcreator={LaTeX via pandoc}}

\title{Titel}
\usepackage{etoolbox}
\makeatletter
\providecommand{\subtitle}[1]{% add subtitle to \maketitle
  \apptocmd{\@title}{\par {\large #1 \par}}{}{}
}
\makeatother
\subtitle{Untertitel}
\author{Max Mustermann \and Maria Musterfrau}
\date{2024-01-29}

\begin{document}
  \begin{frontmatter}
  \begin{titlepage}
  %%%%%%%%%%%%%%%%%%%%%%%%%%%%%%%%%%%%%%%%%
  % Academic Title Page
  % LaTeX Template
  % Version 2.0 (17/7/17)
  %
  % This template was downloaded from:
  % http://www.LaTeXTemplates.com
  %
  % Original author:
  % WikiBooks (LaTeX - Title Creation) with modifications by:
  % Vel (vel@latextemplates.com)
  %
  % License:
  % CC BY-NC-SA 3.0 (http://creativecommons.org/licenses/by-nc-sa/3.0/)
  % 
  % Instructions for using this template:
  % This title page is capable of being compiled as is. This is not useful for 
  % including it in another document. To do this, you have two options: 
  %
  % 1) Copy/paste everything between \begin{document} and \end{document} 
  % starting at \begin{titlepage} and paste this into another LaTeX file where you 
  % want your title page.
  % OR
  % 2) Remove everything outside the \begin{titlepage} and \end{titlepage}, rename
  % this file and move it to the same directory as the LaTeX file you wish to add it to. 
  % Then add \input{./<new filename>.tex} to your LaTeX file where you want your
  % title page.
  %
  %%%%%%%%%%%%%%%%%%%%%%%%%%%%%%%%%%%%%%%%%

  %----------------------------------------------------------------------------------------
  %	PACKAGES AND OTHER DOCUMENT CONFIGURATIONS
  %----------------------------------------------------------------------------------------


  	\newcommand{\HRule}{\rule{\linewidth}{0.5mm}} % Defines a new command for horizontal lines, change thickness here
  	
  	\center % Centre everything on the page
  	
  	\includegraphics{logo.jpeg}
  	
  	%------------------------------------------------
  	%	Headings
  	%------------------------------------------------
  	
  	\textsc{\LARGE Institution Name}\\[1.5cm] % Main heading such as the name of your university/college
  	
  	\textsc{\Large Major Heading}\\[0.5cm] % Major heading such as course name
  	
  	\textsc{\large Minor Heading}\\[0.5cm] % Minor heading such as course title
  	
  	%------------------------------------------------
  	%	Title
  	%------------------------------------------------
  	
  	\HRule\\[0.4cm]
  	
  	{\huge\bfseries An Unnecessarily Convoluted Academic Title}\\[0.4cm] % Title of your document
  	
  	\HRule\\[1.5cm]
  	
  	%------------------------------------------------
  	%	Author(s)
  	%------------------------------------------------
  	
  	\begin{minipage}{0.4\textwidth}
  		\begin{flushleft}
  			\large
  			\textit{Author}\\
  			B.J. \textsc{Blazkowicz} % Your name
  		\end{flushleft}
  	\end{minipage}
  	~
  	\begin{minipage}{0.4\textwidth}
  		\begin{flushright}
  			\large
  			\textit{Supervisor}\\
  			Dr. Caroline \textsc{Becker} % Supervisor's name
  		\end{flushright}
  	\end{minipage}
  	
  	% If you don't want a supervisor, uncomment the two lines below and comment the code above
  	%{\large\textit{Author}}\\
  	%John \textsc{Smith} % Your name
  	
  	%------------------------------------------------
  	%	Date
  	%------------------------------------------------
  	
  	\vfill\vfill\vfill % Position the date 3/4 down the remaining page
  	
  	{\large\today} % Date, change the \today to a set date if you want to be precise
  	
  	%------------------------------------------------
  	%	Logo
  	%------------------------------------------------
  	
  	%\vfill\vfill
  	%\includegraphics[width=0.2\textwidth]{placeholder.jpg}\\[1cm] % Include a department/university logo - this will require the graphicx package
  	 
  	%----------------------------------------------------------------------------------------
  	
  	\vfill % Push the date up 1/4 of the remaining page
  \end{titlepage}
  \end{frontmatter}

  \ifdefined\Shaded\renewenvironment{Shaded}{\begin{tcolorbox}[sharp corners, enhanced, boxrule=0pt, frame hidden, interior hidden, borderline west={3pt}{0pt}{shadecolor}, breakable]}{\end{tcolorbox}}\fi

\renewcommand*\contentsname{Inhaltsverzeichnis}
{
\setcounter{tocdepth}{2}
\tableofcontents
}
\listoffigures
\listoftables
\setstretch{1.5}
\mainmatter
\addcontentsline{toc}{chapter}{\listfigurename} % Füge das Verzeichnis der Abbildungen zum Inhaltsverzeichnis hinzu
\addcontentsline{toc}{chapter}{\listtablename} % Füge das Verzeichnis der Abbildungen zum Inhaltsverzeichnis hinzu

\hypertarget{abstract}{%
\chapter*{Abstract}\label{abstract}}
\addcontentsline{toc}{chapter}{Abstract}

Ab Kapitel Kapitel~\ref{sec-quarto} beginnt der Inhalt.

\hypertarget{sec-quarto}{%
\chapter{Quarto}\label{sec-quarto}}

\hypertarget{introducing}{%
\section{Introducing}\label{introducing}}

Quarto enables you to weave together content and executable code into a
finished document. To learn more about Quarto see
\url{https://quarto.org}.

\hypertarget{running-code}{%
\chapter{Running Code}\label{running-code}}

\hypertarget{first}{%
\section{first}\label{first}}

When you click the \textbf{Render} button a document will be generated
that includes both content and the output of embedded code. You can
embed code like this:

\begin{verbatim}
[1] 2
\end{verbatim}

\hypertarget{second}{%
\section{second}\label{second}}

You can add options to executable code like this

\begin{verbatim}
[1] 4
\end{verbatim}

\hypertarget{spezification}{%
\subsection{spezification}\label{spezification}}

The \texttt{echo:\ false} option disables the printing of code (only
output is displayed).

If yo set \texttt{echo:\ false} in the \texttt{execute} in the yaml it
is set generally\ldots.

\hypertarget{langer-satz}{%
\chapter{Langer Satz}\label{langer-satz}}

Hier ist ein sehr langer Satz, der extrem viele Verschachtelungen hat -
wie diese hier -, um zu zeigen, was passiert, wenn der Satz länger als
eine Zeile ist.

\hypertarget{the-including-quarto-file}{%
\chapter{the including quarto file}\label{the-including-quarto-file}}

\newpage{}

\hypertarget{header-2}{%
\section{Header 2}\label{header-2}}

here is some text here is an reference to {„Hauptseite``} (2023)

\newpage{}

\hypertarget{unterkapitel-2}{%
\section{Unterkapitel 2}\label{unterkapitel-2}}

dsahdgdsaölgh

\hypertarget{include-a-graphic}{%
\chapter{Include a graphic}\label{include-a-graphic}}

Abbildung~\ref{fig-LOGO} zeigt das Logo.

\begin{figure}

{\centering \includegraphics{logo.png}

}

\caption{\label{fig-LOGO}Logo}

\end{figure}

\hypertarget{verweis-auf-literatur}{%
\chapter{Verweis auf Literatur}\label{verweis-auf-literatur}}

Mgl 1: {„Hauptseite``} (2023) sagt dies und das.

Mgl 2: Es gilt dies und das ({„Hauptseite``} 2023).

\hypertarget{make-tables}{%
\chapter{Make tables}\label{make-tables}}

Einfache/individuelle Tabelle (Tabelle~\ref{tbl-letters}):

\hypertarget{tbl-letters}{}
\begin{longtable}[]{@{}lll@{}}
\caption{\label{tbl-letters}My Caption}\tabularnewline
\toprule\noalign{}
Col1 & Col2 & Col3 \\
\midrule\noalign{}
\endfirsthead
\toprule\noalign{}
Col1 & Col2 & Col3 \\
\midrule\noalign{}
\endhead
\bottomrule\noalign{}
\endlastfoot
A & B & C \\
E & F & G \\
A & G & G \\
\end{longtable}

\newpage{}

kable-Tabelle wird in Tabelle~\ref{tbl-example_table} dargestellt.

\hypertarget{tbl-example_table}{}
\begin{longtable}[]{@{}lll@{}}
\caption{\label{tbl-example_table}table caption}\tabularnewline
\toprule\noalign{}
Col1 & Col2 & Col3 \\
\midrule\noalign{}
\endfirsthead
\toprule\noalign{}
Col1 & Col2 & Col3 \\
\midrule\noalign{}
\endhead
\bottomrule\noalign{}
\endlastfoot
1 1 & 1 2 & 1 3 \\
2 1 & 2 2 & 2 3 \\
3 1 & 2 3 & 3 3 \\
\end{longtable}

\newpage{}

test

\newpage{}

test

\newpage{}

\hypertarget{test}{%
\section{test}\label{test}}

test

\hypertarget{formel}{%
\chapter{Formel}\label{formel}}

Hier wird ein Formelbeispiel Gleichung~\ref{eq-qdot} gezeigt:

\begin{equation}\protect\hypertarget{eq-qdot}{}{
\dot{Q} = \dot{m} \cdot c_w \cdot \Delta{T}
}\label{eq-qdot}\end{equation}

\hypertarget{literaturverzeichnis}{%
\chapter*{Literaturverzeichnis}\label{literaturverzeichnis}}
\addcontentsline{toc}{chapter}{Literaturverzeichnis}

\hypertarget{refs}{}
\begin{CSLReferences}{1}{0}
\leavevmode\vadjust pre{\hypertarget{ref-example_wikipedia}{}}%
{„Hauptseite``}. 2023. Wikipedia. Februar 2023.
\url{https://de.wikipedia.org/wiki/Wikipedia:Hauptseite}.

\end{CSLReferences}

\appendix
\renewcommand{\thechapter}{\Alph{chapter}}
\setcounter{chapter}{0}
\counterwithin{figure}{chapter}
\counterwithin{table}{chapter}

\hypertarget{anhang-1}{%
\chapter{Anhang 1}\label{anhang-1}}

\setcounter{page}{1}
\renewcommand{\thepage}{\thechapter.\arabic{page}} % Seitenzahlformat ändern

\hypertarget{unterhang}{%
\section{Unterhang}\label{unterhang}}

\hypertarget{tbl-letters-full}{}
\begin{longtable}[]{@{}llll@{}}
\caption{\label{tbl-letters-full}1st Appendix table}\tabularnewline
\toprule\noalign{}
Col1 & Col2 & Col3 & Col4 \\
\midrule\noalign{}
\endfirsthead
\toprule\noalign{}
Col1 & Col2 & Col3 & Col4 \\
\midrule\noalign{}
\endhead
\bottomrule\noalign{}
\endlastfoot
A & B & C & D \\
E & F & G & H \\
I & J & K & L \\
M & N & O & P \\
Q & R & S & T \\
U & V & W & X \\
Y & Z & - & - \\
\end{longtable}

\newpage{}

test

\newpage{}

test

\hypertarget{anhang-2}{%
\chapter{Anhang 2}\label{anhang-2}}

\setcounter{page}{1}

\hypertarget{tbl-letters2-full}{}
\begin{longtable}[]{@{}llll@{}}
\caption{\label{tbl-letters2-full}2nd Appendix table}\tabularnewline
\toprule\noalign{}
Col1 & Col2 & Col3 & Col4 \\
\midrule\noalign{}
\endfirsthead
\toprule\noalign{}
Col1 & Col2 & Col3 & Col4 \\
\midrule\noalign{}
\endhead
\bottomrule\noalign{}
\endlastfoot
A & B & C & D \\
E & F & G & H \\
I & J & K & L \\
M & N & O & P \\
Q & R & S & T \\
U & V & W & X \\
Y & Z & - & - \\
\end{longtable}

\newpage{}

test

\newpage{}

test


\backmatter

\end{document}
